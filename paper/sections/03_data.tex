%!TEX root = ../paper.tex
\section{Data Preparation}
\label{sec:data}

As mentioned in Section~\ref{sec:related}, there are many parameters and design decisions in the data generation and preparation process, which influence the final performance.
This section discusses all these parameters in detail.
In general we first extract the frame images from a video, and then calculate optical flow images based on two adjacent frames.
This captures the movement between those two frames.
After the data generation, the preprocessed data has to be converted into a suitable data file for the Caffe framework.
In the following section we will discuss the different ways and data formats available for each step in detail.

\subsection{Dataset}
For our research purposes we relied on the UCF101~\cite{soomro2012ucf101} dataset, a popular choice in the computer vision community.
UCF101 is an action recognition data set of realistic action videos, collected from YouTube, having 101 action categories.
The videos in 101 action categories are grouped into 25 groups, where each group can consist of 4-7 videos of an action.
The videos from the same group may share some common features, such as similar background, similar viewpoint, etc.
The action categories can be divided into five types:
\begin{enumerate*}
	\item Human-Object Interaction
	\item Body-Motion Only
	\item Human-Human Interaction
	\item Playing Musical Instruments
	\item Sports.
\end{enumerate*}
The dataset contains of 13320 clips with a fixed frame rate and a resolution of \texttt{320 x 240} respectively.
There are a few videos in the \emph{PommelHorse} category, which have a different resolution of \texttt{400 x 226}.
We dealt with these by rescaling them to \texttt{320 x 240}.
On average the clips have a length of 7 seconds.

For comparable results the dataset's authors published three fixed train/test splits.
We worked on the first split, i.e. \texttt{trainlist01.txt} and \texttt{testlist01.txt}\footnote{ \url{http://crcv.ucf.edu/data/UCF101/UCF101TrainTestSplits-RecognitionTask.zip}}.
When working with UC101 it is important to follow these splits, because training instances are often sampled and cut from a longer video sequence.
Concretely, this means that dataset's authors gathered long activity video sequences and cut them into short clips to create more than one training sample from this sequence.
It is important to keep the samples from one original sequence in the same dataset or otherwise the test performance is estimated too high.

\subsection{Frame extraction}
The first preprocessing step needs to convert the given \texttt{*.avi} video files into single-frame pictures.
We extracted the frame data from the videos with the \emph{FFmpeg}\footnote{\url{http://www.ffmpeg.org}} tool.
See Section~\ref{subsec:frame_extraction} for the concrete parameters.
As output, we chose JPEG files.
Finding the correct frame rate for the frame extraction was challenging.
High frame rates, such as 30 frames per second, often lead to two adjacent frames being exactly identical.
This is a problem for the optical flow extraction, since there will be almost no measurable difference between the images and the optical flow is reported as empty.
Especially for classes, where a lot of movement and characteristic optical flow is expected (such as \emph{Archery} or \emph{Juggling Balls}), this turned out to be problematic.
The problem of identical images does not exist for lower frame rates, e.g. 5 frames per second.
However in this case we create less overall training data and less details, especially for optical flow extraction.
A variable frame rate extraction is not feasible, as the optical flow must be comparable between different video types, i.e. the time between two frames must be identical.
This is why we decided to use a fixed frame rate of 15 frames per second.
% COMMENT: The data on the server is definitely 15 frames per second. There are twice as much frames in the frames_30fps/ subfolders than in the frames/ subfolders.
This minimized the occurrence of two identical adjacent frames while still giving a sufficient amount of detail.

\subsection{Optical flow extraction}
Optical flow is computed to capture the movement in a video sequence and is always based on two immediate consecutive grey-scale frames.
We used the optical flow algorithm from Brox et al~\cite{brox2004high}, an accepted standard in the research community.
We were able to apply to out of the box OpenCV algorithm \footnote{\url{http://docs.opencv.org/modules/gpu/doc/video.html\#gpu-broxopticalflow}} and benefited from its GPU computation, leading to faster flow extraction.
Optical flow can be computed both along X and Y axis. Therefore there are two optical flow images for each pair of consecutive frames, resulting in $\frac{N - 1}{2}$ optical flows for $N$ frames.

\todo{Figure}

\subsection{Optical flow frame stacking}
A single optical flow image does not contain as much information as the RGB picture of the same frame.
Therefore, several optical flow images are stacked, just as the color channels \emph{R}, \emph{G}, and \emph{B} are stacked in a standard image.
For each frame, we stack the optical flows for the next 10 frames in both x and y direction, leading to a total stack size of 20.
We experimented with adding the flows of the 10 previous frames as well, leading to a stack size of 40.
However, this did not improve the results.
Note, stacking is done with a sliding window. E.g. we add the optical flows from the \nth{1} to the \nth{2} frame, from the \nth{2} to the \nth{3} frame, and so on. X and Y flows are interleaved.


\subsection{Data format for caffe}
% Experiences with LMDB, LevelDB, HDF5
The caffe framework accepts different formats for the input data. Commonly the data is stored in a database of the type LMDB, HDF5, or LevelDB.
In our initial tests, we used LMDB, as this is usually used in the Caffe examples, and seems to be the recommended option.
Using LMDB resulted in very large databases, because it does not support compression.
This led to disk space issues.

We went on to use HDF5, which compresses the data.
However, we relied on Python library h5py\footnote{\url{http://www.h5py.org/}}, which turned out to have major memory leaks. Therefore we cannot recommend using HDF5 in Python environment.
The library was consuming so much RAM and swap space, that the machine had to be killed at times.
Also, the Caffe documentation states, that HDF5 should only be used ``when efficiency is not critical''.

Finally we settled on LevelDB, a database developed by Google for fast read performance.
Compression is done using Google's Snappy\footnote{\url{http://google.github.io/snappy/}} library, which offers both good performance and compression.
We therefore recommend LevelDB. A complete comparison of the DB creation time and size can be found in table \ref{table:databases}.

\todo{Who has the table values?}
\begin{table}[]
\centering
\caption{Data format comparison}
\label{table:databases}
\begin{tabular}{lll}
\toprule
Creation Time 		& DB Size  & DB Type \\ \midrule
TODO           & XX GB  	 & LMDB \\
TODO          		& XX GB  	 & HDF5 \\
TODO          		& XX GB  	 & LevelDB \\
\bottomrule
\end{tabular}
\end{table}

\subsection{Image cropping}
As mentioned before, most images have a resolution of \texttt{320 x 240}.
On the other hand, most convolutional network architectures require a squared input resolution.
The \emph{Caffenet} requires data to be in \texttt{227 x 227}, the \emph{CNN\_M} network requires a resolution of \texttt{224 x 224}.
The best approach to transform an image from its original resolution to the required resolution is cropping.
Caffe already comes with a built-in method for randomly cropping images.
When setting the \texttt{crop\_size} parameter in the net definition, the input image is cropped randomly every time the image is read for training or testing.
Also, mirroring can be used to further increase the training size with the same approach.

Some authors~\cite{ye2015evaluating} perform this cropping manually, by taking a clipping from each corner and from the center of the image.
We also tried this approach, but discarded it from our final setup, because of two reasons.
Firstly, it massively increases the training set size making the data harder to handle and requiring more disk space.
Secondly, the random cropping in Caffe is at least as effective as manually cropping and can save us some work.

\subsection{Selecting frames for the LSTM}
For the LSTMs, we need another set of databases (spatial and flow), which contain only a fixed number of frames per video.
We decided to use 16 frames per video, on the grounds that the LSTM implementation \footnote{\url{https://github.com/jeffdonahue/caffe/tree/recurrent}} by Jeff Donahue cannot handle arbitrarily long input sequences, especially over batches.
From every videos' frame pool we selected these 16 frames by omitting the first and last five frames, and the equally distributing the selected frames among the remaining frames.
By skipping some frames at the beginning and end, we avoided the fade-in/fade-out effects in some of the videos. To create such a selection see section \ref{subsec:listfile_to_file_ucf_static} for details.

\subsection{Relevant scripts}
\todo{Which of the script are still relevant?}

\begin{itemize}
\item build\_flow\_lmdb.sh
\item crop\_mirror\_frames.py
\item frame\_extraction.sh
\item frame\_extraction\_force\_overwrite.sh
\item frames\_to\_file\_ucf\_flow.py
\item frames\_to\_file\_ucf\_flow\_cropped.py
\item frames\_to\_file\_ucf\_static-de
\item frames\_to\_hdf5.py
\item listfile\_to\_file\_ucf\_static.py
\item remove\_duplicate\_frame.py
\item resize.py
\item sce\_frames.sh
\end{itemize}

\subsubsection{caffe/tools/convert\_imageset}
The script \texttt{convert\_imageset} is a script provided by Caffe and bundles individual images into either a LMDB or LevelDB.
The database type can be switched with the \texttt{backend} parameter.

\begin{lstlisting}[language=sh, caption=convert\_image.sh, label=lst:convert_imageset]
# Usage
caffe/tools/convert_imageset [FLAGS] ROOTFOLDER/ LISTFILE DB_NAME

# Example
caffe/tools/convert_imageset -backend Leveldb imageroot filelist testdatabase.leveldb
\end{lstlisting}

\subsubsection{caffe/tools/convert\_imageset\_multi}
\label{subsec:convert_imageset_multi}
The script \texttt{convert\_imageset\_multi} is an addition to  \texttt{convert\_imageset} developed by us. It can be used to stack a number of images from the file list into single multi-channel image.
The \texttt{STACK\_SIZE} parameter specifies how many images will be merged into one.
Internally, it uses the method \texttt{cv::merge} from OpenCV to stack the images before writing them to the database.
\begin{lstlisting}[language=sh, caption=convert\_image\_multi.sh, label=lst:convert_imageset]
# Usage
caffe/tools/convert_imageset_multi [FLAGS] ROOTFOLDER/ LISTFILE DB_NAME STACK_SIZE

# Example
caffe/tools/convert_imageset_multi -backend Leveldb imageroot filelist testdatabase.leveldb 16 
\end{lstlisting}

\subsubsection{tools/frame\_extraction.sh}
\label{subsec:frame_extraction}
The script \texttt{tools/frame\_extraction.sh} is responsible for extracting the image frames from all UCF101 video files using \emph{FFmpeg}.
The parameters passed to \emph{FFmpeg} are as follows: \lstinline{ffmpeg -n -nostdin -i "$file" -r "$framerate" -qscale:v 2 "$outputfolder/%3d.jpg"}.
\todo{Joseph}
\begin{lstlisting}[language=sh, caption=convert\_image\_multi.sh, label=lst:convert_imageset]
# Usage
tools/frame\_extraction.sh [FLAGS] VIDEOFOLDER/ OUTPUTFOLDER/ FRAMERATE

# Example
tools/frame\_extraction.sh /opt/data_sets/UCF-101/videos/ /opt/data_sets/UCF-101/frames_10fps/ 10
\end{lstlisting}

\subsubsection{tools/listfile\_to\_file\_ucf\_static.py}
\label{subsec:listfile_to_file_ucf_static}